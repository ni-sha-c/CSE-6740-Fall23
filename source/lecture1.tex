\documentclass[final]{beamer}
\usepackage{eulervm,verbatim}          
\usepackage[scaled]{helvet}
\usepackage[most]{tcolorbox}
\setbeamercolor{frametitle}{fg=black,bg=white} % Colors of the block titles
\setbeamertemplate{caption}{\raggedright\insertcaption\par}
\setbeamertemplate{caption}{\raggedright\insertcaption\par}
\definecolor{darkcerulean}{rgb}{0.03, 0.27, 0.49}
\newcommand{\citesmall}[1]{[{\color{darkcerulean}\begin{small} \textbf{#1} \end{small}}]}
\setbeamertemplate{footline}[frame number]
\DeclareMathOperator*{\argmin}{arg\,min}
\usepackage{graphicx}  % Required for including images

\usepackage{booktabs} % Top and bottom rules for tables
\definecolor{burgundy}{rgb}{0.5, 0.0, 0.13}
\newcommand{\highlight}[1]{{\color{burgundy} \textbf{#1}}}
\usepackage{hyperref}
\hypersetup{
    colorlinks=true,
    linkcolor=blue,
    filecolor=magenta,      
    urlcolor=magenta,
    pdftitle={Course Syllabus},
    pdfauthor={Nisha Chandramoorthy},
    pdflang={en-US}
}



%----------------------------------------------------------------------------------------
%	TITLE SECTION 
%----------------------------------------------------------------------------------------
\title{\begin{huge}{CSE 6740 A/ISyE 6740: Computational Data Analysis: Introductory lecture}\end{huge}} % Poster title


\author{Nisha Chandramoorthy} % Author(s)


%----------------------------------------------------------------------------------------

\begin{document}

\frame{\titlepage}

%----------------------------------------------------------------------------------------
%	OBJECTIVES
%----------------------------------------------------------------------------------------
\begin{frame}{Course logistics}
\begin{itemize}
	\item Instructor: Nisha Chandramoorthy. 
	\pause
	\item Interested in teaching and learning about foundations of machine learning
	\pause
	\item 6 TAs: Akpevwe Ojameruaye, Atharva Ketkar, Chengrui Li, Darryl Jacob,, Mithilesh Vaidya, and Yusen Su 
	\pause
	\item Lectures: TR 12:30-1:45 pm. OH: 30 minutes after. Location: East Architecture 123.
	\pause
	\item Grade: 4 homeworks (30\%), 2 midterms (30\%), final project (40\%)
	\pause
\item \href{https://canvas.gatech.edu}{Canvas} (see syllabus), \href{https://www.gradescope.com/courses/578036}{gradescope}, \href{https://piazza.com/gatech/fall2023/cse6740a/info}{Piazza}, \href{https://github.com/ni-sha-c/CSE-6740-Fall23}{Github}
\end{itemize}
\end{frame}

\begin{frame}{Machine learning and data mining: what are they?}
\begin{itemize}
	\item ``Automated detection of meaningful patterns in data'' - Shalev-Shwartz and Ben-David.
	\pause
	\item  Goal in this class: understand the foundations (``why''s and ``how''s) of ML
	\pause
	\item ML = Compute + data
	\pause
	\item Compute: Optimization, representation/models
	\pause
	\item Data: distributions, features/compression, statistics
\end{itemize}
\end{frame}
\begin{frame}{Categorizations of learning}
	\begin{itemize}
		\item Supervised, unsupervised, self-supervised, semi-supervised 
		\pause
	\item Supervised: using \emph{experience} (training data) to learn  
	\pause
	\item Unsupervised: using \emph{data} to identify patterns, match distributions?
	\pause
	\item Mode of learning and testing are different 
	\end{itemize}
\end{frame}
\begin{frame}{Research frontier in understanding}
	\begin{itemize}
		\item how to make and test conjectures about how large language models (LLMs) learn?
		\pause
		\item ``how to train them better (more efficiently)'' -- number of practical questions perhaps benefit from theory

	\end{itemize}
\end{frame}
\begin{frame}{History - I}
\end{frame}
\begin{frame}{Recent breakthroughs}

\end{frame}
\begin{frame}{Tools for the ``why'' and ``how'' questions}

\end{frame}
\begin{frame}{Supervised learning framework}
\end{frame}
\begin{frame}{PAC learning}
\end{frame}
\begin{frame}{Linear regression}


\end{frame}
\end{document}
