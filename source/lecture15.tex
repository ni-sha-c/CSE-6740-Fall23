\documentclass[final]{beamer}
\usepackage{eulervm,verbatim}          
\usepackage[scaled]{helvet}
\usepackage[most]{tcolorbox}
\setbeamercolor{frametitle}{fg=black,bg=white} % Colors of the block titles
\setbeamertemplate{caption}{\raggedright\insertcaption\par}
\setbeamertemplate{caption}{\raggedright\insertcaption\par}
\definecolor{darkcerulean}{rgb}{0.03, 0.27, 0.49}
\newcommand{\citesmall}[1]{[{\color{darkcerulean}\begin{small} \textbf{#1} \end{small}}]}
\setbeamertemplate{footline}[frame number]
\DeclareMathOperator*{\argmin}{arg\,min}
\usepackage{graphicx}  % Required for including images
\usepackage{bbm}
\usepackage{booktabs} % Top and bottom rules for tables
\definecolor{burgundy}{rgb}{0.5, 0.0, 0.13}
\newcommand{\highlight}[1]{{\color{burgundy} \textbf{#1}}}
\usepackage{hyperref}
\hypersetup{
    colorlinks=true,
    linkcolor=blue,
    filecolor=magenta,      
    urlcolor=magenta,
    pdftitle={CSE6740-Lecture 14},
    pdfauthor={Nisha Chandramoorthy},
    pdflang={en-US}
}



%----------------------------------------------------------------------------------------
%	TITLE SECTION 
%----------------------------------------------------------------------------------------
\title{\begin{huge}{Lecture 14: Neural Networks}\end{huge}} % Poster title


\author{Nisha Chandramoorthy} % Author(s)


%----------------------------------------------------------------------------------------

\begin{document}

\frame{\titlepage}

%----------------------------------------------------------------------------------------
%	OBJECTIVES
%----------------------------------------------------------------------------------------
\begin{frame}{Last time}
	\begin{itemize}
		\item A hypothesis class restricted to some $C$ \emph{shatters} $C$ if any binary function on $C$ is in the class
		\pause
		\item VC dimension, ${\rm VCdim}(\mathcal{H})$: maximal size of $C \subseteq \mathcal{X}$ that is shattered by $\mathcal{H}.$ 
		\pause
		\item Eg 1. VCdim of threshold functions on $\mathbb{R}$ is 1.
		\pause
		\item Eg 2: VCdim of indicator functions on intervals of $\mathbb{R}$ is 2.
		\pause
		\item Eg 3: VCdim of a finite class $\mathcal{H}$ $\leq \log_2|\mathcal{H}|$ 
	\end{itemize}
\end{frame}
\begin{frame}{Generalization bounds based on VC dimension}
	\begin{itemize}
		\item $\mathcal{H} = \{h_\theta(x) = \sin(\theta x): \theta \in \mathbb{R}\}$.
		\pause
		\item VCdim is $\infty$
		\pause 
	\item Binary classification generalization for 0-1 loss over class $\mathcal{H}$ with VCdim = $d$: there exist constants $C_1, C_2 > 0$ such that 
		$$ C_1 \dfrac{d + \log(1/\delta)}{\epsilon^2} \leq m_{\mathcal{H}}(\epsilon, \delta) \leq C_2 \dfrac{d + \log(1/\delta)}{\epsilon}$$
	
	\end{itemize}

\end{frame}
\begin{frame}{(partial) History - trace back from transformers (source:Wikipedia)}
	\begin{itemize}
		\item Transformer architecture: 2017, Google Brain [Vaswani et al]
		\item Deep learning, unsupervised learning 2010s (e.g., GANs 2014)...
		\item ImageNet: 2009, Fei Fei Li 
		\item Long-short term memory (LSTM) architecture: 1997, [Hochreiter and Schmidhuber]
		\item Convolutional NNs: (inspired from) 1979 work by [Fukushima];  Recurrent neural networks: 1982 [Hopfield]
		\item ...
		\item Automatic Differentiation: 1970 [Linnainmaa]
		\item ...
		\item First neural networks: 1950s [Minsky and others]
	\end{itemize}
\end{frame}

\begin{frame}{Fully connected Neural Networks}
	\begin{itemize}
		\item Neuron: input $\sum_j w_j h_j$; output $\sigma(\sum_j w_j h_j)$
		\item Organized into layers of depth $l$ and width $n$
		
		\item Graph: $V, E, \sigma, w$; weight function.
		\pause
	\item VC dim of $\mathcal{H}_{V, E, {\rm sign}} \leq C |E|\log |E|$
		\pause
		\begin{itemize}
		\item Proof: Growth function $\tau_{\mathcal{H}}(m) = \max_{C, |C|=m} |\mathcal{H}|_C| $
			\pause Fact 1: Let $\mathcal{H} =\mathcal{H}_l \circ \cdots \circ \mathcal{H}_1.$ Then,  $\tau_{\mathcal{H}}(m) \leq \prod_{t=1}^l \tau_{\mathcal{H}_t}(m)$.
			\pause 
			\item
			Fact 2: Let $\mathcal{H} =\mathcal{H}^{(1)}  \cdots \circ \mathcal{H}^{(n)}.$ Then,  $\tau_{\mathcal{H}}(m) \leq \prod_{t=1}^l \tau_{\mathcal{H}^{(t)}}(m)$.
			\pause 
		\item Fact 3: Sauer's Lemma: $\tau_{\mathcal{H}}(m) = (em/d)^d$, where $d \geq $ VCdim $(\mathcal{H})$ 
		\end{itemize}
	\end{itemize}

\end{frame}
\begin{frame}{Training}

\begin{tcolorbox}
	SGD update step: $w_{t+1} = w_t - \eta \tilde{\nabla}\hat{R}_S(w_t)$
\end{tcolorbox}
	\begin{itemize}
		\item $\tilde{\nabla}\hat{R}_S(w_t)$ is an unbiased estimate of $\nabla \hat{R}_S(w_t)$
		\item Convergence for convex $\hat{R}_S$ and $\eta$ small enough.
		\item Gradients implemented using backpropagation algorithm
	\end{itemize}
\end{frame}

\begin{frame}{Universal approximation theorems}
	\begin{tcolorbox}
		Theorem [Park et al 2020, ICLR] (Informal) For $f\in L^p(\mathbb{R}^n, \mathbb{R}^m)$, and any $\epsilon > 0$, there exists a fully connected ReLU network $F$ of width exactly $d = \max\{n+1,m\}$ such that $\|f - F\|_p^p < \epsilon.$
	\end{tcolorbox}
	\begin{tcolorbox}
		Kolmogorov-Arnold-Sprecher representation theorem: Any continuous multivariate function $f:\mathbb{R}^n\to\mathbb{R}$ can be written as $$f(x) = \sum_{i=0}^{2n} \Phi(\sum_{j=1}^n w_j\sigma(x_i+\eta i) + i),$$ where $\sigma:[0,1]\to[0,1].$
	\end{tcolorbox}
\end{frame}
\begin{frame}{Convolutional Neural Networks (source: cs231n.stanford.edu)}
	\begin{itemize}
		\item Suitable for image recognition. Won the 2012 ImageNet competition and subsequent ones.
		\item Three types of layers: convolutional, FC, pooling
		\item Convolutional layer: accepts a volume of size $W_1 \times H_1 \times D_1$ and outputs a volume of size $W_2 \times H_2 \times D_2$ where $W_2 = (W_1 - F + 2P)/S + 1$ and $H_2 = (H_1 - F + 2P)/S + 1$ and $D_2 = K$.

		\item $K$ is number of filters, $F$ is filter size, $S$ is stride, $P$ is padding.
		\item Pooling layer: downsamples along width and height, and optionally along depth.
		\item FC layer: computes class scores, resulting in volume of size $1 \times 1 \times K$.

	
	\end{itemize}
\end{frame}
\end{document}
